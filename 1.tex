%\renewcommand{\contentsname}{CONTENTS}
%\renewcommand{\bibname}{Bibliography}
%\renewcommand{\indexname}{INDEX}
%\renewcommand{\figurename}{図 }
%\renewcommand{\tablename}{表 }
%\renewcommand{\appendixname}{APPENDIX}

\renewcommand{\part}{Contents}
\renewcommand{\prechaptername}{第 }
\renewcommand{\postchaptername}{ 章}

\chapter{序論}
 
本章では、最初に本研究における背景およびその現状の問題点を述べる。
そのあと、これに対する本研究の目的とアプローチについて述べる。
そして最後に、本論文の構成について示す。

\section{背景}
\subsubsection{IoT}
IoTとは、物理空間の様々なモノがネットワークに繋がり、そのデータに基づいて組織の意思や他のモノの動きが決定される世界の概念を表す言葉である。
特にこの一連の流れの際、人間が意図的にデータ入力をしたりデータ送信をしたりする必要がなく、これらをモノが自発的に人間にとってはシームレスに行うことをIoTという言葉で表す。
そしてこのIoTは我々の生活に大きな恩恵をもたらしている。
例えば既に販売されているサービスとして存在するものとして、道路事業者や交通事業者向けにその会社の自動車のGPS情報を取得し、交通情報を提示するものがある。\cite{hitachi_traffic}
これは、道路事業者が利用者に対する利便性の向上を、交通事業者が業務の効率化を測れるようにするものである。
また、自宅の外に温度センサ取り付けることでピンポイントで温度や湿度が取得でき、その情報をスマートフォンでスマートフォンから閲覧できる製品がある。\cite{therometer}
これにより、屋外に出ることなく手元のデバイスですぐ外の気温を確認でき、例えば屋内で今日の服装を決定することができる。
このように、我々はIoTによって様々な利益を得ている。

この便利なIoTであるが、これの思想に基づいてサービスやアプリケーションを作り上げるには、コストのかかる工程が大きく分けて3つ存在する。
1つ目はSensing、情報を取得する必要がある。
交通情報の例では、各事業者の車にGPSを設置する部分がこれに当たる。
また、もしある交通事業者が直近に通っていない交通区間があったとすると、その区間の交通情報を取得することはできない。
温度計センサの例では、自分の家のすぐ外に温度計を設置する部分がこれに当たる。
2つ目はProcessing、情報を処理する必要がある。
交通情報の例では、GPSから取得した位置情報があまり変わっていないようであればそこが渋滞している可能性があると判断することがこれに当たる。
温度計の例では、特定の温度範囲を逸脱した場合、スマートフォンへ通知を送るようになっている部分がこれに当たる。
3つ目はActuation、情報を活かして行動する必要がある。
交通情報の例では、渋滞情報を地図上にマッピングしてわかりやすく提示することがこれに当たる。
温度計の例では、スマートフォンやPC上に温度を表示することがこれに当たる。
なお、ここで挙げた二つの例ではどちらもディスプレイに表示することがActuationに当たるが、他にも「工場内で温度上昇を検知した場合、工場内の生産機器の稼働率を下げる」ということを自動で行うこともこのActuationに当たる。
以上の流れは一般にSPA(Sensing、Processing、Actuationの略)と称され、これらを経てIoTの様々な製品やサービスは構築される。

\subsubsection{IoTデータ市場}
ところで、現状はこれらを全て一つの主体が行う必要がある。
これら全てでIoTサービスが出来上がるので、当然と言えば当然だ。
しかし近年、これはIoTデータを売買できるプラットフォームである「IoTデータ市場」と呼ばれるものが出現している。
その中の一つがEverySense\cite{everysense}だ。
EverySenseはIoTデータを売買できるプラットフォームである。
このIoTデータ市場について、先の交通情報の例を使い、様々な観点から考えてみよう。
最初に、IoTデータの買い手の視点に立つ。
同じ道路を走る車にいくつものGPSセンサを取り付ける必要は、企業間の垣根を取り払えば存在しない。
同じ道路に同一事業者の車がいないので、その区間の交通情報を取得するためにセンサを取り付ける必要があるのだ。
もし他の会社の車のGPS情報を買い、取得することが出来れば、わざわざGPSセンサを取り付ける必要はない。
更に、先ほどは自社の車が通っていない交通区間についての情報を取得することはできなかったが、情報を買うことができれば通っていない道の交通情報も分かる。
次に、IoTデータの売り手の視点に立つ。
今まではGPSセンサを取り付けることは自社の為のみであった。
したがって、GPSセンサの代金や取り付けの工事費は全て自社のコストとなり、そのコストは顧客の払った売り上げから賄っていた。
しかしGPSセンサのデータが売れることが分かれば、このコストの一部はデータの買い手が負担することになり、価格面で顧客サービス向上につながる。
最後に、全体を俯瞰する観点に立つ。
同じ時刻に同じ場所を走行する別事業者の車両が1台ずつ、計2台が存在していたとする。
片方の会社はもう片方の会社から車両データを買えば良いので、IoTデータ市場の出現によって無駄なGPSセンサが1台減ることとなる。
更に、IoT化を進める上で不可欠なセンサが物理空間に増える可能性を秘めているのだ。
データの売り手がデータ取得の費用が全て既存の顧客が払った売り上げから賄うわけではないと分かった場合、更に多くのセンサを車両に取り付ける可能性がある。
この時、世の中全体で使えるIoTセンサ量は増加し、世の中全体のIoT化が今までより容易に進むようになる。
このように、様々なステークホルダに利益をもたらし得るのがこのIoTデータ市場である。

また、IoTの3プロセスであるSPAにおいて、このような市場が最も存在しやすいのはこのSensingの段階の市場である。
その理由をEverySenseのCEOである眞野浩氏は次のように述べている。\cite{everysense_ceo}
「AGFA(Apple,Google,Facebook,Amazon)に代表されるように,従来のインターネットは,社会や自然に対して個々の企業や個人が観察や実験,日々の活動から得た情報を分析,蓄積して得た知見を製品化し,流通させ,展開させる分野に利用されてきた。ここで扱われる情報は,単純なデータと比べると,フィルタリング,組み合わせ,解析などの処理により高い付加価値が与えられている。このような高付加価値の情報は,価値とともに生成コストも高価ゆえに,生成者の所有欲が深くなり,流動性を損なう。逆に,上流工程の処理を伴わない,いわゆるRawデータほど,付加価値が低く,生成者の流通に対する許容性は高い。」
他の段階では高い付加価値の加工後データを触るので他者への許容性は低いが、生データは低い付加価値であるので流通への許容性が高いとしている。
このようにして、IoT市場が作られつつある。