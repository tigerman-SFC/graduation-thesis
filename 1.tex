%\renewcommand{\contentsname}{CONTENTS}
%\renewcommand{\bibname}{Bibliography}
%\renewcommand{\indexname}{INDEX}
%\renewcommand{\figurename}{図 }
%\renewcommand{\tablename}{表 }
%\renewcommand{\appendixname}{APPENDIX}

\renewcommand{\part}{Contents}
\renewcommand{\prechaptername}{第 }
\renewcommand{\postchaptername}{ 章}

\chapter{序論}
 
本章では、最初に本研究における背景およびその現状の問題点を述べる。
そのあと、これに対する本研究の目的とアプローチについて述べる。
そして最後に、本論文の構成について示す。

\section{背景}
私たちの身の回りには様々なIoT製品が存在している。
その最たる例はスマートフォンであろう。
Google Now\cite{google_now}は生活の中において、必要な情報を聞く前に教えてくれる技術である。
例えば、夜遅くまで外にいるとき、ユーザがスマホに聞くことなく終電の時間を教えてくれる機能がある。
これはスマートフォンのGPS機能と、現在時刻、交通機関のダイヤを参照した上で通知を与えている。
他にも、ウェアラブルデバイスが注目されている。
fitbit\cite{fitbit}は腕時計式のウェアラブルデバイスである。
アプリをインストールすると、デバイスから取得した歩行数や心拍数、睡眠時間、食事、消費カロリーなどのデータを閲覧できる。
他にも、車にカメラを取り付けることで道路上の白線の掠れを検知し、塗り直すべき白線の箇所を取得する研究\cite{dragnman_hakusen}がある。
これによって、今までは別途調査が必要であった道路の白線の掠れている場所の検知が簡単になった。
このように、IoT製品\UTF{00B7}サービスは様々な利益を我々に与えてくれている。
そしてこれらのIoT製品\UTF{00B7}サービスは全て取得したIoTデータから我々に有益な情報を与えてくれているのだ。
この元データなしにIoTの製品\UTF{00B7}サービスは決して生まれない。
そこでこのIoTデータの流動性を高めるため、IoTデータ市場というものが近年、考えられている。
その市場ではIoTデータを事業者間で売買できるようになっていて、取引の際の手数料をこの市場を管理する管理者へ払うようになっている。
他にも、この市場に参加する際や、参加し続ける際に管理者へ払うようになっている制度も存在する。
このように一定の仲介手数料は存在するものの、IoTデータをより簡単に調達できるようになるIoTデータ市場は、買い手にとって利益をもたらしてくれるものである。
またこのIoTデータ市場は売り手にとっても、今まで自社でしか活用用途のなかったデータを販売することが可能になる点で、利益を得られる。
このように、IoTデータ市場は買い手と売り手の双方にとって利益を享受することのできるものであるため、これからIoT市場全体の成長に伴って出現\UTF{00B7}発展していくものと考えられている。

\section{IoTデータ市場に関する問題}
この便利なIoT製品\UTF{00B7}サービスを支えるIoTデータの元となり得るIoTデータ市場であるが、今の状況は中央集権的なサービスである。
一般に、中央集権的な管理者がいるサービスと管理者のいないサービスが存在し、各々で良い点と悪い点が存在する。
管理者のいるサービスの良い点は、秩序を整えることが容易なこと、サービスのシステムがシンプルに保てることである。
例えばサービスの上で倫理的に悪いことをしているユーザを発見したとする。
この際、管理者がそのユーザのアカウントを凍結するなど、秩序を乱す原因となるものを素早く取り除くことができる。
また、新しい便利な機能を追加する時に、管理者がスピード感を持ってデプロイできる。
自分以外に合意を取るプロセスが不必要であるためであり、これも秩序が整っていることの証左である。
そして管理者のいるサービスは、サービスを提供する側とサービスを消費する側ではっきりと役割が分かれており、一般的なサービス形態であることからノウハウも溜まっている。
よってサービスのシステムをシンプルに保つことができるのだ。
一方、管理者がいないサービスの良い点は、そのサービスの透明性が担保されていることである。
管理者がいるサービスにおいて、ユーザに対して不当に悪い扱いをしたとしても、その事実が表出しない。
例えばオンラインショッピングサービスの上で倫理的にも法律的にも正しく、商売を行っていたユーザがいるとする。
だがそのユーザの商売が、管理者も同じショッピングサービス場で行なっている商売と同じであり、管理者の利益を逼迫していたとする。
この時、管理者は悪い行いをしたユーザと同じようにそのユーザのアカウントを凍結することができる。
そしてこの事実を訴えたとしても、技術的にはアカウント凍結を覆すことはできない。
しかし、管理者のいないサービスがこの事態を引き起こすことはない。
なぜなら全てのサービス利用者が対等な立場でサービスを利用するため、特定の利用者が大きな力を得ることがないためである。

さて、今回注目するIoTデータ市場は管理者のいるサービスといないサービスのどちらで運営されるべきであろうか。
IoTデータ市場は参加者にとって商機の存在するサービスであり、サービスを使うことに対して必死になる動機が十分にある。
このように参加者が必死になるサービスにおいて特定の管理者がいた場合、この管理者の持つ権限は巨大なものとなる。
例えばA社とB社がライバル関係であり、どちらも管理者のいる同じIoTデータ市場で商売をしていたとする。
この時、A社はB社に不利になるような、あるいはA社にとって有利になるような仕様の改変を求める可能性は十分にある。
更に管理者はその改変が出来てしまうので、仕様の改変を行う代わりに管理者がA社に対して見返りを求めることが可能になる。
この見返りを求められるような力こそが、管理者の持つ巨大な力の正体である。
そしてこのような状態は市場全体に対して良い影響を与えない。
自社のコストカットによる値段の抑制を行う、あるいは他者との差別化を図れるデータ収集を行って商機を図るといった正しい競争原理が働かなくなる。
権力者に媚びることが利益を上げる最善の手法となってしまうためだ。
更にIoTデータ市場は巨大な市場となりうるため、その権力者は更に大きな力を持ち、見返り目的に行動しやすくなる。
このような危険性は、市場という公正で透明な取引が必要とされる場所において致命的である。
ここから分かるように、市場は公正で透明な取引の担保が必要であるのだ。
一方で、IoTデータ市場に中央集権の管理者が存在することの利点を考えてみるが、市場はこれから多くの新機能を必要とされる種類のサービスではないだろう。
また、倫理的に悪いことをしているユーザに関しても特定の団体が排除するべきものではなく、国や参加者の総意によって排除されるべきものである。
特定の管理者がユーザを排除できることは市場においては逆にリスクである。

以上のように考え、市場のようなお金の絡むシステムに関しては管理者がいないほうが望ましい。
しかし、現状は管理者の存在するIoTデータマーケットしか存在しない。
本研究ではこれが問題であると考える。

\section{目的とアプローチ}
本研究の先にある最終の目的は、利用者にとって自由の割合と支配の割合が最も理想的な塩梅で、市場の統治が徹頭徹尾透明なIoTデータ市場を構築することである。
具体的に述べると、最終目的にある市場は利用者の意思や希望と関係なく、技術的に仕方なく存在していたIoTデータ市場の巨大な管理者という存在を利用者にとって必要な分のルールや管理団体よって透明な形で提供するものである。
そこで本研究では現実世界の市場と同じように、一つの組織にも囚われない、ルールのみによって縛られるオープンなIoTデータ市場を作ることを目的とする。
そして本研究の先に、利用者にとって必要なルールや必要とされる管理団体が存在するIoTデータ市場が構築される取り組みが存在するべきである。
この手法を取る理由は、今までの手法を発展させてより管理団体の権限を減らすことは難しいが、本研究で提示する手法を発展させて管理団体やルールを作ることは可能であるためである。
そして本研究のように中央集権を無くす際、管理者のいない中での合意アルゴリズムが必要となるが、これにはブロックチェーンを使用する。
また、データの買い手と売り手の間でのデータ通信が必要となるが、これにはメッセージングシステムを使用する。
この二つを統合させ、IoTデータ市場を作り出すことが本研究のアプローチである。

\section{本論文の構成}
本論文は本章を含めて8章からなる。
本章ではIoTが我々の生活の役に立っていることと、そのためにはデータが不可欠でその市場が誕生していること、しかしそこには管理者がいるという問題点が存在することを示した。
また、それに対する目的とアプローチを述べた。
2章ではこれをさらに詳細に、技術的な観点も含めて論じる。
3章ではブロックチェーン技術について簡単に述べ、今回使用するEthereumやオフチェーン技術について触れる。
4章では提案するIoTデータ市場の機能要件およびそのプラットフォーム上での取引の流れを述べる。
5章では提案する市場に関して、設計と実装を述べる。
6章では提案する市場に関して、トランザクション流通量などの定量評価を行う。
7章では今後の展望について、ブロックチェーン技術の観点と社会的な観点から論じる。
8章では本論文のまとめを述べる。