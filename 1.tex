%\renewcommand{\contentsname}{CONTENTS}
%\renewcommand{\bibname}{Bibliography}
%\renewcommand{\indexname}{INDEX}
%\renewcommand{\figurename}{図 }
%\renewcommand{\tablename}{表 }
%\renewcommand{\appendixname}{APPENDIX}

\renewcommand{\part}{Contents}
\renewcommand{\prechaptername}{第 }
\renewcommand{\postchaptername}{ 章}

\chapter{序論}
 
本章では、最初に本研究における背景およびその現状の問題点を述べる。
そのあと、これに対する本研究の目的とアプローチについて述べる。
そして最後に、本論文の構成について示す。

\section{背景}
私たちの身の回りには様々なIoT製品が存在している。
その最たる例はスマートフォンであろう。
Google Now\cite{google_now}は生活の中において、必要な情報を聞く前に教えてくれる技術である。
例えば、夜遅くまで外にいるとき、ユーザがスマホに聞くことなく終電の時間を教えてくれる機能がある。
これはスマートフォンのGPS機能と、現在時刻、交通機関のダイヤを参照した上で通知を与えている。
他にも、ウェアラブルデバイスが注目されている。
fitbit\cite{fitbit}は腕時計式のウェアラブルデバイスである。
アプリをインストールすると、デバイスから取得した歩行数や心拍数、睡眠時間、食事、消費カロリーなどのデータを閲覧できる。
他にも、車にカメラを取り付けることで道路上の白線の掠れを検知し、塗り直すべき白線の箇所を取得する研究\cite{dragnman_hakusen}がある。
これによって、今までは別途調査が必要であった道路の白線の掠れている場所の検知が簡単になった。
このように、IoT製品\UTF{00B7}サービスは様々な利益を我々に与えてくれている。
そしてこれらのIoT製品\UTF{00B7}サービスは全て取得したIoTデータから我々に有益な情報を与えてくれているのだ。
この元データなしにIoTの製品\UTF{00B7}サービスは決して生まれない。
そこでこのIoTデータの流動性を高めるため、IoTデータ市場というものが近年、考えられている。
その市場ではIoTデータを事業者間で売買できるようになっていて、取引の際の手数料をこの市場を管理する管理者へ払うようになっている。
他にも、この市場に参加する際や、参加し続ける際に管理者へ払うようになっている制度も存在する。
このように一定の仲介手数料は存在するものの、IoTデータをより簡単に調達できるようになるIoTデータ市場は、買い手にとって利益をもたらしてくれるものである。
またこのIoTデータ市場は売り手にとっても、今まで自社でしか活用用途のなかったデータを販売することが可能になる点で、利益を得られる。
このように、IoTデータ市場は買い手と売り手の双方にとって利益を享受することのできるものであるため、これからIoT市場全体の成長に伴って出現\UTF{00B7}発展していくものと考えられている。

\section{IoTデータ市場に関する問題}
この便利なIoT製品\UTF{00B7}サービスを支えるIoTデータの元となり得るIoTデータ市場であるが、ここには問題が存在する。
問題とは、管理者が存在することだ。
この管理者の存在が、市場全体を不健康な状態へと導く。
詳細には、管理者が好き勝手に市場全体をコントロール出来るので、この管理者に敵対する組織はこの市場に入れない或いは入ったとしても利益が出にくいような制約を受けてしまう可能性がある。
また国の市場とは異なり、公正な取引がなされているかを監視するインセンティブが管理者に存在せず、公正取引が実現されない可能性がある。
また公正取引が実現されなかった場合、偽データでの詐欺などがあった場合でも、それをこのプラットフォーム上で罰することが行われない可能性も存在する。
このように、現在の管理者の存在するIoTデータ市場には大きな問題点が存在する。

\section{目的とアプローチ}
そこで、本研究では管理者の存在しないIoTデータ市場を提案、実装することを目的とする。
この際、管理者のいない中での合意アルゴリズムが必要となるが、これにはブロックチェーンを使用する。
また、データの買い手と売り手の間でのデータ通信が必要となるが、これにはメッセージングシステムを使用する。
この二つを統合させ、IoTデータ市場を作り出すことが本研究のアプローチである。

\section{本論文の構成}
本論文は本章を含めて8章からなる。
本章ではIoTが我々の生活の役に立っていることと、そのためにはデータが不可欠でその市場が誕生していること、しかしそこには管理者がいるという問題点が存在することを示した。
また、それに対する目的とアプローチを述べた。
2章ではこれをさらに詳細に、技術的な観点も含めて論じる。
3章ではブロックチェーン技術について簡単に述べ、今回使用するEthereumやオフチェーン技術について触れる。
4章では提案するIoTデータ市場の機能要件およびそのプラットフォーム上での取引の流れを述べる。
5章では提案する市場に関して、設計と実装を述べる。
6章では提案する市場に関して、トランザクション流通量などの定量評価を行う。
7章では今後の展望について、ブロックチェーン技術の観点と社会的な観点から論じる。
8章では本論文のまとめを述べる。